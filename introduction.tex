\section{Introduction}

The rapid advancements in networking have necessitated the development of many newer protocols and packet processing functionalities. The difficulty in programming and extension of traditional hardware routers has made software routers --packet processing software running on general purpose hardware- more popular. In effect, software routers have realized the modification of network functionality with a software upgrade, avoiding the burden of hardware developers in designing and developing newer hardware.

The evolution of networks has also increased the length, complexity, and diversity of the code in software routers. Thus programming in low-level languages like C++ is equivalent to reinventing the wheel consuming time and effort of the programmer. The business of manual optimizations in the code further complicates the job of a programmer. For example, Routebricks achieved 24Gbps of IPv4 throughput on two quad-core Nehalem processors. But the authors had to do very careful manual tuning to achieve such performance. In spite of that, the other workloads deviating from the setup might take a significant performance hit.

Many Domain Specific Languages like P4\cite{Bosshart:2014:PPP:2656877.2656890}, Click\cite{kohler2000click} etc. are introduced with design features aimed at abstracting the packet processing pipelines. The real benefit of using the DSLs come with designing the compilers to map programs to target routers that exploit platform specific optimizations thus alleviating the programmer's burden by offloading the performance tuning. For example, in Power of Batching\cite{Kim:2012:PBC:2349896.2349910}, Click router, a popular DSL has been optimized to achieve 28 Gbps of IPv4 throughput on two quad-core Intel Xeon X5550 CPUs i.e., 3.5 Gbps per core. In the context of P4C, a prototype compiler for P4 language, is able to generate a program that achieves 5.17 Gbps IPv4 throughput per core on Intel XEON E5-2630. On the other hand the hand optimised DPDK example is able to achieve 10Gbps per core on the same machine. The performance difference highlights the efficiency gap between compiler generated code and handtuned code.

The key issues in software routers are due to the overheads in memory stalls and redundant operations like function calls that waste CPU cycles. On Intel x86 platforms, memory access latency is 90-130 ns. It is often the case that ineffective coding styles cause more memory accesses than necessary thus increasing CPU cycles per packet.  In addition, multiple packets during processing pass through the same function flow. Thus there is a benefit in amortizing the overheads in function calls. In this work, we leverage batching as an efficient means to hide memory stalls and amortize redundant operations. We also demonstrate forwarding performance improvements by adding batching capabilities to the P4C compiler. The primary contributions of this paper are:

\begin{enumerate}
\item Developed a platform agnostic mathematical formulation that allows one to investigate the benefits of batching theoretically.
\item Extended the P4C compiler for the Intel x86 platform to generate router code. Our compiler generates batched code of the target router.
\item Detailed performance evaluation and comparison of popular network applications like L2fwd, IPv4, IPv6, NDN with other popular manually optimized applications.
\end{enumerate}
