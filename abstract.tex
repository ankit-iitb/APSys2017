\begin{abstract}
Software packet processing is increasingly commonplace, especially
for software-defined networking constructs. Previous work has
investigated methods to efficiently map packet processing pipelines
to general-purpose processor architectures. Concurrently, novel
high-level domain-specific
languages (DSLs) for specifying modern packet processing
pipeline functionality, are emerging (e.g., P4 \cite{Bosshart:2014:PPP:2656877.2656890}).
An attractive goal is develop
a compiler that can automatically map a high-level
pipeline specification (specified in a high-level DSL) to an
underlying machine architecture. Ideally, the compiler should automatically
exploit the available parallelism, make intelligent scheduling decisions,
and adapt to the workload needs in an online fashion, to provide maximum
performance. An important pre-requisite for the
development of such a compiler is a
performance model of the underlying machine architecture, for
the applications of interest.

We report our experiences with adding an optimizer
to the P4C compiler \cite{Laki:2016:HSP:2934872.2959080}, which
compiles a high-level P4 program to a lower-level C-based implementation that runs
with the DPDK infrastructure \cite{DPDK}, and gets eventually executed
on a multi-socket x86 machine. We make two contributions: (a) we show
that significant performance improvements (up to 55\%) can be gained by adding
scheduling and prefetching
optimizations to the P4C compiler; and (b) we develop a performance
model for reasoning about
the expected throughput and latency of a packet-processing workload, on a
modern machine architecture. Our model can be used by a compiler,
to reason about the expected
performance of a packet-processing workload for different code configurations, and
can thus be used to optimize the generated code accordingly.
\end{abstract}
